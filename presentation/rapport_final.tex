\documentclass[twocolumn,a4paper]{IEEEtranfr}
%\documentclass
% verbatim : pour afficher le contenu d'un fichier 
%
\usepackage{verbatim}
% gère les graphiques .jpg, .png ,\dots.
\usepackage[dvips]{graphicx}
%
% Pour la gestion de la couleur 
%
\usepackage{color}

%
% Formule chimique
%
\usepackage{chemfig}
%
% + de maths avec AMS
%

\usepackage{amsmath}
\usepackage{amssymb}
%
% Pour afficher des algorithmes
%
\usepackage{algorithm}
\usepackage{algorithmic}
\usepackage{listings}
%
% pour l'hypertexte
%
\usepackage{url}
\usepackage{hyperref}
% français
\usepackage[french]{babel}
% accents
%
% ucs 
% utf8x 
% 
\usepackage{ucs}
\usepackage[utf8x]{inputenc}
%\usepackage[T1]{fontenc}
\usepackage{subfigure}
\usepackage{framed}

\makeatother
% Placer vos figures et images dans les répertoires suivants
% Ne jamais mettre les noms de 
%
% Attention le / final est important 
\graphicspath{{./images/}{./figures/}{../presentation/images/}}

% alias
%
% A consommer sans modération
%
%

\newcommand{\blst}{\begin{lstlisting}}
\newcommand{\elst}{\end{lstlisting}}
\newcommand{\beqan}{\begin{eqnarray*}}
\newcommand{\eeqan}{\end{eqnarray*}}
\newcommand{\beqa}{\begin{eqnarray}}
\newcommand{\eeqa}{\end{eqnarray}}
\newcommand{\bear}{\begin{eqnarray}}
\newcommand{\ear}{\end{eqnarray}}
\newcommand{\bears}{\begin{eqnarray*}}
\newcommand{\ears}{\end{eqnarray*}}
\newcommand{\beq}{\begin{equation}}
\newcommand{\eeq}{\end{equation}}
\newcommand{\rref}[1]{(\ref{#1})}
\newcommand{\eref}[1]{\rref{#1}}
\newcommand{\gd}{\stackrel{.}{\geq}}
\newcommand{\ld}{\stackrel{.}{\leq}}
%%\newcommand{\eqref}[1]{(\ref{#1})}
\renewcommand{\r}{\right}
\renewcommand{\l}{\left}
\newcommand{\lbr}{\left \{ }
\newcommand{\rbr}{\right \} }
\newcommand{\Lbr}{\left [}
\newcommand{\Rbr}{\right ]}
\newcommand{\lp}{\left (}
\newcommand{\rp}{\right )}
%\newcommand{\mylabel}[1]{\label{#1}  \mbox{~~ \tiny \bf [ #1 ] } }
\newcommand{\mylabel}{\label}

%% Notations d'ensembles
\newcommand{\X}{{\cal X}}
\newcommand{\Y}{{\cal Y}}
%\newcommand{\C}{{\cal C}}
\newcommand{\D}{{\cal D}}
\renewcommand{\S}{{\cal S}}
\newcommand{\T}{{\cal T}}
\newcommand{\R}{{\cal R}}
\renewcommand{\H}{{\cal H}}
\newcommand{\V}{{\cal V}}
\renewcommand{\P}{{\cal P}}

%% variables
\newcommand{\eps}{\epsilon}
\newcommand{\real}{{\mathcal {R}}}
\newcommand{\complex}{{\mathcal {C}}}

%% scalaires
\newcommand{\err}{\mathcal{E}}
\newcommand{\dmin}{D_{\min}}
\newcommand{\dl}{D_\ell}
\newcommand{\tw}{\tilde{w}}
\newcommand{\tx}{\tilde{x}}
\newcommand{\ty}{\tilde{y}}
\newcommand{\ha}{h^a}
\newcommand{\dftd}{\tilde{d}}
\newcommand{\dftw}{\tilde{w}}
\newcommand{\dfty}{\tilde{y}}
\newcommand{\dfth}{\tilde{h}}
\newcommand{\Lagrange}{{\mathcal L}}
\newcommand{\Ntones}{N_c}
\newcommand{\hk}{h^{(k)}}
\newcommand{\rh}{{\tt h}}
%%\renewcommand{\ell}{l}

%% vecteurs
\newcommand{\vR}{{\bf R}}
\newcommand{\vmu}{\mbox{\boldmath$\mu$}}
\newcommand{\vbreve}[1]{\v{#1}}
%\renewcommand{\v}[1]{{\bf #1}}
\newcommand{\xmmse}{\hat{\vx}_{{\rm mmse}}}
\newcommand{\va}{{\bf a}}
\newcommand{\vone}{{\bf 1}}
\newcommand{\vb}{{\bf b}}
\newcommand{\vm}{{\bf m}}
\newcommand{\vw}{{\bf w}}
\newcommand{\vwul}{{{\bf w}_{\rm ul}}}
\newcommand{\wul}{{{w}_{\rm ul}}}
\newcommand{\vwdl}{{{\bf w}_{\rm dl}}}
\newcommand{\wdl}{{{w}_{\rm dl}}}
\newcommand{\vs}{{\bf s}}
\newcommand{\vy}{{\bf y}}
\newcommand{\vyul}{{{\bf y}_{\rm ul}}}
\newcommand{\yul}{{{y}_{\rm ul}}}
\newcommand{\pul}{{{P}_{\rm ul}}}
\newcommand{\vydl}{{{\bf y}_{\rm dl}}}
\newcommand{\ydl}{{{y}_{\rm dl}}}
\newcommand{\pdl}{{{P}_{\rm dl}}}
\newcommand{\vz}{{\bf z}}
\newcommand{\vr}{{\bf r}}
\newcommand{\vc}{{\bf c}}
\newcommand{\vh}{{\bf h}}
\newcommand{\vg}{{\bf g}}
\newcommand{\vx}{{\bf x}}
\newcommand{\vxul}{{\bf x}_{\rm ul}}
\newcommand{\xul}{{x}_{\rm ul}}
\newcommand{\vxdl}{{\bf x}_{\rm dl}}
\newcommand{\xdl}{{x}_{\rm dl}}
\newcommand{\vd}{{\bf d}}
\newcommand{\ve}{{\bf e}}
\newcommand{\vv}{{\bf v}}
\newcommand{\vt}{{\bf t}}
\newcommand{\vu}{{\bf u}}
\newcommand{\vP}{{\bf P}}
\newcommand{\vq}{{\bf q}}
\newcommand{\vp}{{\bf p}}
%\newcommand{\vg}{{\bf g}}
\newcommand{\vxN}{{\bf x}^{\bf N}}
\newcommand{\vyN}{{\bf y}^{\bf N}}
\newcommand{\tvw}{{\bf \tilde{w}}}
\newcommand{\tvx}{{\bf \tilde{x}}}
\newcommand{\tvy}{{\bf \tilde{y}}}
\newcommand{\tty}{y'}
\newcommand{\vxa}{{\bf x}^{\bf a}}
\newcommand{\vya}{{\bf y}^{\bf a}}
\newcommand{\vwa}{{\bf w}^{\bf a}}
\newcommand{\var}{{\bf e}_{\bf r}}
\newcommand{\vat}{{\bf e}_{\bf t}}
\newcommand{\vD}{{\bf D}}
\newcommand{\vY}{{\bf Y}}
\newcommand{\vW}{{\bf W}}
\newcommand{\vdftd}{{\bf \tilde{d}}}
\newcommand{\vdftw}{{\bf \tilde{w}}}
\newcommand{\vdfty}{{\bf \tilde{y}}}
\newcommand{\vdfth}{{\bf\tilde{h}}}
\newcommand{\dftmH}{{\bf \tilde{H}}}
\newcommand{\vdftx}{{\bf \tilde{x}}}
\newcommand{\vdfta}{{\bf \tilde{a}}}
\newcommand{\vxA}{{\bf x}^{\bf A}}
\newcommand{\vxB}{{\bf x}^{\bf B}}
\newcommand{\vxAl}{{\bf x}^{\bf A}_{\bf \ell}}
\newcommand{\vxBl}{{\bf x}^{\bf B}_{\bf \ell}}
\newcommand{\rl}{r^{(\ell)}}
\newcommand{\wl}{w^{(\ell)}}
%% matrices
\newcommand{\mQ}{{\bf Q}}
\newcommand{\mU}{{\bf U}}
\newcommand{\mV}{{\bf V}}
\newcommand{\mPsi}{{\bf \Psi}}
\newcommand{\mUt}{{\bf U}_t}
\newcommand{\mUr}{{\bf U}_r}
\newcommand{\mX}{{\bf X}}
\newcommand{\mLambda}{\mathbf{\Lambda}}
\newcommand{\mF}{{\bf F}}
\newcommand{\mK}{{\bf K}}
\newcommand{\mG}{{\bf G}}
\newcommand{\mA}{{\bf A}}
\newcommand{\mB}{{\bf B}}
\newcommand{\mC}{{\bf C}}
\newcommand{\mD}{{\bf D}}
\newcommand{\mR}{{\bf R}}
\newcommand{\mH}{{\bf H}}
\newcommand{\mHa}{{\bf H^a}}
\newcommand{\mI}{{\bf I}}
\newcommand{\mk}{{\bf K}}
\newcommand{\mv}{{\bf V}}
\newcommand{\mO}{{\bf O}} %% orthogonal 
\newcommand{\mJ}{{\bf J}} %% pseudo covariance
\newcommand{\rH}{{\tt H}}
%%\newcommand{\dftmH}{\tilde{\mH}}
\newcommand{\sm}[1]{\sum_{#1=-\infty}^{+\infty}}
\newcommand{\smr}[3]{\sum_{#1=#2}^{#3}}
\newcommand{\smu}[1]{\sum_{#1=1}^{+\infty}}
\newcommand{\smz}[1]{\sum_{#1=1}^{+\infty}}
\newcommand{\ejm}[1]{e^{-j\omega #1}}
\newcommand{\ejp}[1]{e^{j\omega #1}}
\newcommand{\usdp}{\frac{1}{2\pi}}
\newcommand{\edjm}[1]{e^{-j 2\pi f #1}}
\newcommand{\edjp}[1]{e^{j 2\pi f #1}}
%% math notation
\newcommand{\sinc}{{\rm sinc}}
\newcommand{\CN}{\mathcal{CN}}
\newcommand{\N}{\mathcal{N}}
\newcommand{\indistrib}{\stackrel{\mathcal{D}}{\rightarrow}} 
\newcommand{\inprob}{\stackrel{\mathcal{P}}{\rightarrow}}   
\newcommand{\mc}[1]{\mathcal{#1}}

%
%
%  Début du document 
%
%
\begin{document}

\title{Etude et mise en place d'Edge Node sur la base d'OpenEdgeComputing}
\author{HOANG Tuan Dung, KAF Merwan, LE CORRE Pierre} 

% place le titre 
\maketitle

\begin{abstract}
En vue de la multiplication future des objets connectés, la bande passante du réseau internet risque d'être saturé par la trop grande demande de ressources faites aux serveurs distants du Cloud. Afin de palier à ce problème, Orange Labs nous a demandé d'étudier une des solutions existantes : Open Edge Computing. Cette solution consiste à se servir de la puissance de calculs des objets en périphérie du réseau (typiquement une livebox ou encore, un routeur 5G) afin de permettre aux objets connectés d'accèder à de la puissance de calculs sans remonter jusqu'au serveurs distants. Il nous a ensuite été demandé de nous pencher sur l'implémentation d'une telle solution et d'en éstimer la viabilité. Cela permettrait à Orange d'avancer sur leur recherche d'Edge Node.
\end{abstract} 

\begin{keywords}
Orange SA, réseau 5G, Edge Node, open source, Open Edge Computing, Open Stack, Multicast Domain Name System, Service Discovery
\end{keywords}

%\markboth{This is for left pages}{and this is for right pages}


\section{Introduction}

Notre projet s'intègre au sein d'Orange Labs à Cesson-Sévigné. Notre projet se trouve dans le début de leur projet qui consiste à étudier et rechercher des réseaux de cinquième génération du futur marché. En effet, les recherches sur le futur des réseaux ont déjà été menés par les chercheurs américains en collaboration avec des géants de services informatiques et télécommunications. Notre projet part sur le résultat des chercheurs américains que la technologie Open Edge Computing est un choix viable. Notre projet a donc un enjeu majeur de permettre à Orange de savoir la viabilité d'Open Edge Computing, ainsi qu'une base de code à débuter la phase de développement.

\section{Méthode} 

Notre projet s’est déroulé en 3 phases distinctes : 

\begin{itemize}
\item Recherche sur edge computing et plus spécifiquement sur le projet open source OpenEdgeComputing;
\item Mise en place du projet OpenEdgeComputing;
\item Mise en place de notre propre solution.
\end{itemize}

\subsection{Recherche sur OpenEdgeComputing}

La première partie de notre projet consistait à l’étude des différents projets EdgeComputing, et plus précisément celui de OpenEdgeComputing. Nous avons étudié l’architecture général des différents projets EdgeComputing (cf. figure représentant l’architecture).

\subsection{Mise en place de OpenEdgeComputing}

blbabla


\subsection{Mise en place de notre solution}

\subsubsection{LOLOLO}


blabal

\section{Résultats}

\subsection{Recherche sur OpenEdgeComputing}

blabl

\subsection{Mise en place de OpenEdgeComputing}

blabl

\subsection{Mise en place de notre solution}

blabl

blalbla

\section{Analyse}

blalba

\section{Discussion}

blabla

\section{Conclusion}
Il faut toujours écrire une conclusion. 

\section{Références bibliographiques}

%\\bibliographystyle{IEEEbib} %numérique
%\bibliographystyle{utphys}
%\bibliographystyle{alpha}  %alphanumérique
%\\bibliography{bibliographie}  



\end{document}
